\documentclass[a4paper,10pt]{article}

\usepackage[francais]{babel} 
\usepackage[utf8]{inputenc}

\usepackage{amsfonts}
\usepackage{amsmath}
\usepackage{amssymb}
\usepackage{mathtools}
\usepackage{textcomp}
\usepackage{stmaryrd}

\usepackage{tikz}
\usetikzlibrary {positioning}

\usepackage{listings}
\usepackage{url}
\usepackage{graphicx}

\usepackage{listings}
\usepackage{xcolor}

\textwidth 173mm \textheight 235mm \topmargin -50pt \oddsidemargin
-0.45cm \evensidemargin -0.45cm

\newcommand{\eqdef}{\stackrel{\mathclap{\normalfont\small\mbox{def}}}{=}}
\newcommand{\code}[1]{{\fontfamily{pcr}\selectfont #1}}

\newcommand{\htodo}[1]{\begin{huge}\colorbox{yellow}{\textcolor{red}{\textbf{TODO~:} #1}}\end{huge}}
\newcommand{\todo}[1]{\colorbox{yellow}{\textcolor{red}{\textbf{TODO~:} #1}}}

\newcommand{\ldblbrack}{\text{\textlbrackdbl}}
\newcommand{\rdblbrack}{\text{\textrbrackdbl}}

\title{Optimisation combinatoire et convexe~: projet voyageur de commerce}
\author{Nathanaël Courant et David Saulpic}
\date{}

\begin{document}

\maketitle

\section{Commentaires généraux}

\section{Résultats sur TSPLIB}

\section{Réponses aux questions}

Même avec les contraintes supplémentaires sur les coupes, le programme linéaire n'est pas entier~: en effet, considérons le graphe suivant~:

\begin {center}
\begin {tikzpicture}[,auto ,node distance =2 cm and 3cm ,on grid ,
semithick ,
state/.style ={ circle ,top color =white , bottom color = white ,
draw,, minimum width =1 cm}]
\node[state] (C) {$1$};
\node[state] (A) [above left=of C] {$0$};
\node[state] (B) [above right =of C] {$2$};
\node[state] (D) [below=of C] {$3$};
\node[state] (E) [below left=of D] {$4$};
\node[state] (F) [below right=of D] {$5$};
\path (C) edge node {$1 | 1/2$} (A);
\path (C) edge node {$1 | 1/2$} (B);
\path (B) edge node {$1 | 1/2$} (A);
\path (C) edge node {$0 | 1$} (D);
\path (A) edge node {$0 | 1$} (E);
\path (B) edge node {$0 | 1$} (F);
\path (E) edge node {$0 | 1/2$} (D);
\path (F) edge node {$0 | 1/2$} (E);
\path (D) edge node {$0 | 1/2$} (F);
\end{tikzpicture}
\end{center}

Les annotations $a|b$ sur les arêtes siginifient que le coût de l'arête est de $a$, et que la variable $x_e$ correspondante a pour valeur $b$ dans la solution optimale du programme linéaire. On s'aperçoit que la valeur optimale de ce programme linéaire est donc de $3/2$~: il ne peut donc pas être entier.

On peut également remarquer que pour toute coupe $S$ de $V$, et pour toute solution entière $x$ du problème initial, $\sum_{e\in\delta(S)} x_e$ est paire. En particulier, si $|\delta(S)|$ est impair, $\sum_{e\in\delta(S)} x_e \leq |\delta(S)| - 1$. On vient donc d'obtenir une classe d'inégalités linéaires satisfaites par toute solution entière du problème, mais pas par les solutions fractionnaires comme on a pu le constater avec l'exemple ci-dessus.

\end{document}
